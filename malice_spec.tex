\documentclass[a4wide, 10pt]{article}
\usepackage{a4, fullpage}
\usepackage{graphicx}
\setlength{\parskip}{0.3cm}
\setlength{\parindent}{0cm}
\newcommand{\tab}{\hspace*{2em}}

\begin{document}

\title{MAlice Language Specification}

\author{Sean Allan \and Mitchell Allison \and Max Tottenham}

\date{\today}         % inserts today's date

\maketitle            % generates the title from the data above

\section{Introduction}
The goal of this document is to outline an informal language specification for 
the programming language MAlice. The following sections will outline the rough
grammatical structure of the language in Extended-BNF (EBNF) \cite{Wikipedia},
followed by a description of the language structure and semantics.

\section{BNF Grammar} 
\begin{verbatim}
    Function    = "The looking-glass hatta" ,  "()" ,  "opened" , 
                   Codeblock , "closed" ;
    LogExp      = LogExp , '&' , Exp |  LogExp , '|' , Exp 
                | LogExp , '^' , Exp | Exp ;
    Exp         = Exp , '+' , Term | Exp , '-' , Term | Term ;
    Term        = Term , '*' , Factor | Term , '/' , Factor | Factor ;
    Factor      = '~' , Value | Value ;
    Value       = Integer | Identifier ;
    Type        = "number" | "letter" ;
    Declare     = "was a" , Type ;
    Assign      = "became" , Exp | "became" , Char ;
    Return      = "said Alice" ;
    BuiltInFunc = "Ate" | "Drank" ;
    Code        = Identifier , Declare | Identifier , Assign | Exp , Return  
                | Identifier , BuiltInFunc ;
    Codeblock   = Code , Separator , Codeblock | Code , Terminator ; 
    Separator   = Terminator | ',' | "and" | "but" | "then"; 
    Identifier  = String ;
    Terminator  = "too." | '.' | ',' ;
    String      = { Char } ;
    Integer     = '0' | [ '-' ] , Nat ;
    Nat         = DigitNoZero , { Digit } ;
    Digit       = '0' | DigitNoZero ; 
    DigitNoZero = '1' | '2' | '3' | '4' | '5' | '6' | '7' | '8' | '9' ;
    Char        = 'a' | 'b' | 'c' | 'd' | 'e' | 'f' | 'g' | 'h' | 'i' | 'j' | 'k' 
                | 'l' | 'm' | 'n' | 'o' | 'p' | 'q' | 'r' | 's' | 't' | 'u' | 'v'
                | 'w' | 'x' | 'y' | 'z' | 'A' | 'B' | 'C' | 'D' | 'E' | 'F' | 'G'
                | 'H' | 'I' | 'J' | 'K' | 'L' | 'M' | 'N' | 'O' | 'P' | 'Q' | 'R'
                | 'S' | 'T' | 'U' | 'V' | 'W' | 'X' | 'Y' | 'Z' | '_' ;
\end{verbatim}
\section{Semantics}
\subsection{Structure}
The MAlice language primarily describes a program with a single function 
that serves as a point of entry. The format of the function below can be
compared to the '\texttt{main}' function seen in languages such as C and Java.

\begin{verbatim}
    The looking-glass hatta ()  //Function entry point
\end{verbatim}

The function is then enclosed within a block, denoted by the syntax below.
\begin{verbatim}
    opened
        \\code
    closed
\end{verbatim}

The code between the \emph{opened} and \emph{closed} blocks consist of 
declaration and assignment statements followed by a print statement, 
often in this respective order although this isn't strictly enforced. The 
structure of an example code block is shown in Table~\ref{tab:code}.

\begin{table}[h]
\begin{tabular}{l l}
\verb|   meaningoflife was a number.| & \verb|//Declaration| \\
\verb|   meaningoflife became 42.| & \verb|//Assignment| \\
\verb|   meaningoflife said Alice.| & \verb|//Print| \\
\end{tabular}
\caption{Example of code from the MAlice language.}
\label{tab:code}
\end{table}
  
Note that each statement is separated by either a \emph{',', '.', 'too.',
 'and'} or \emph{'but'}. A block of code must be terminated with either a 
 \emph{'.'} or \emph{'too.'}.

\subsection{Data Types}

\subsubsection{\texttt{number}}
\label{sec:number}
{\bf Range:} -128 to +127 {\bf Memory required:} 8 bits.
 
\tab This data type is similar to the \texttt{int} data type from other 
programming languages, or a subset of integers from mathematics. In the MAlice 
language, \texttt{number}s are represented in 8 bit two's complement - this
means they have a range from -127 to +128. The choice to have \texttt{number}s
represented in two's complement was derived from the sample functions whereby
all \texttt{number}s used fell within the -128 to +127 range. Additionally, when
operations such as bitwise NOT were performed, the result suggested that the 
\texttt{number}s were represented in two's complement format.

\subsubsection{\texttt{letter}}

{\bf Range:} a-z, A-Z {\bf Memory required:} 8 bits.

\tab This data type is similar to the \texttt{char} data type from other
programming languages, representing a letter or character. Note that this type
is somewhat restricted in the range of characters that can be represented; 
MAlice only supports lower and upper-case letters (a-z and A-Z). This range was 
deduced from the sample functions whereby no numbers or other characters were 
used in \texttt{letter}s. A \texttt{letter} is stored in 8 bits using ASCII 
representation. The allocation of 8 bits for a \texttt{letter} also allows for
potential future expansion for other characters, such as numbers and other 
symbols (for example, \$).

Also note that the MAlice language does not support a \texttt{string} data 
type, as MAlice does not support arrays (and hence could not store an array of
characters or \texttt{letter}s).

\subsection{Operators}
The operators contained in the MAlice language are described in the below table.
Note that in the table, when a \texttt{number} is given as an acceptable type,
anything (for example, an expression) that evaluates to a number is also
acceptable. In the case of \texttt{ate} and \texttt{drank}, the value contained
within the \texttt{identifier} must evaluate to a \texttt{number}.

The precedence levels are given in ascending order, and all operators are left
recursive. Left recursion is preferable to right recursion in this instance as,
although both can be used in an LALR grammar, the left recursive method often
leads to a smaller stack depth during the evaluation stage of the parser.
\cite{eac}

\begin{tabular}{|l|l|l|l|l|} \hline
Operator & Description & Argument Types & Examples & Precedence Level \\ \hline
$\texttt{ate}$ & Increments the value of an identifier 
& $\texttt{identifier ate}$ & $\texttt{x ate}$ & 1 
\\ \hline

$\texttt{drank}$ & Drecrements the value of an identifier 
& $\texttt{identifier drank}$ & $\texttt{x drank}$ & 1 \\ \hline

$|$ & Performs bitwise OR & $\texttt{number $|$ number}$ & $\texttt{3 $|$ 5}$ 
& 2 \\ \hline

$\land$ & Performs bitwise XOR & $\texttt{number $\land$ number}$ 
& $\texttt{3 $\land$ 5}$ & 2 \\ \hline

$\&$ & Performs bitwise AND & $\texttt{number $\&$ number}$ & $\texttt{3 $|$ 5}$ 
& 2 \\ \hline

+ & Performs arithmetic addition & $\texttt{number + number}$ & $\texttt{3 + 5}$
 & 3 \\ \hline

- & Performs arithmetic subtraction & $\texttt{number - number}$ 
& $\texttt{3 - 5}$ & 3 \\ \hline

* & Performs arithmetic multiplication & $\texttt{number * number}$ 
& $\texttt{3 * 5}$ & 4 \\ \hline

/ & Performs arithmetic division & $\texttt{number / number}$ 
& $\texttt{3 / 5}$ & 4 \\ \hline

\% & Performs arithmetic modulo & $\texttt{number $\%$ number}$ 
& $\texttt{3 $\%$ 5}$ & 4 \\ \hline

$\mathtt{\sim}$ & Performs logical NOT & $\mathtt{\sim}$ $\texttt{number}$
 & $\mathtt{\sim}$ 3 & 5 \\ \hline

\end{tabular}

\subsection{Illegal Operations}

\subsubsection{Divide By Zero}
A divide by zero operation is valid under MAlice syntax, although the
semantics will cause the program to throw a runtime error.

\subsubsection{Numerical operators}
As MAlice is a strongly typed language, numerical operators can only have 
arguments of type \texttt{number}. Any other type used with a numerical operator
produces a compile time 'type error'. This can be seen in the output for 
ex02.alice:

\texttt{Type clash for binary operator '+' on line 5. One type is 'number' and 
the other is 'letter'.}

\subsubsection{Integer Overflow}
As integers are represented in an 8 bit 'two's complement', format there is 
a high chance for integer overflow to occur. Overflow will not be caught
by the compiler. It is up to the implementation on how to handle integer 
overflow at runtime. This language spec document will treat integer 
overflow as undefined behaviour and will ignore it where necessary.

\subsubsection{Variable Declaration}
Variables in MAlice are declared by the BNF rules as above. The MAlice 
language is statically typed, so once a variable has been declared in a
MAlice program, it cannot then be redeclared. Similarly, you cannot introduce
a variable without first declaring its type. The following example is
invalid MAlice syntax:

\texttt{\tab wrong was a number.\\ \tab wrong was a letter.}

The following is also invalid due to a redeclaration of a variable:

\texttt{\tab wrong was a number then wrong was a number.} 

Syntactic errors such as this are caught at compile time.

\subsubsection{Assignment}
Assignment follows the strict typing of MAlice, so a variable of type 
\texttt{letter} cannot take on a value of type \texttt{number}.

\subsubsection{Built in function}
Eating Drinking and talking: 

\begin{thebibliography}{9}
\bibitem{Wikipedia}
\begin{verbatim}	http://en.wikipedia.org/wiki/Extended_Backus%E2%80%93Naur_Form \end{verbatim}


\bibitem{eac}
  Keith D. Cooper and Linda Torczon,
  \emph{Engineering a Compiler}, section 3.5.4.
  Morgan Kaufmann, Massachusetts,
  2nd Edition,
  2012.
\end{thebibliography}


\end{document}


