\documentclass[a4wide, 10pt]{article}
\usepackage{a4, fullpage}
\setlength{\parskip}{0.3cm}
\setlength{\parindent}{0cm}
\newcommand{\tab}{\hspace*{2em}}
% This is the preamble section where you can include extra packages etc.

\begin{document}

\title{MAlice Language Specification}

\author{Sean Allan \and Mitchell Allison \and Max Tottenham}

\date{\today}         % inserts today's date

\maketitle            % generates the title from the data above

\section{Introduction}
The goal of this document is to outline an informal language specification for 
the programming language MAlice. The following sections will outline the rough
gramatical structure of the language in Extended-BNF (EBNF). Then the 
document will explain the semantics, or meaning, of the logical and numerical 
operaters in malice.
\section{BNF Grammar} 
\begin{verbatim}
    - Malice Language Structure - 
    Function    = String ,  "()" ,  "opened" , Codeblock , "closed" ;
    LogExp      = LogExp , '&' , Exp |  LogExp , '|' , Exp | LogExp , '^' , Exp | Exp ;
    Exp         = Exp , '+' , Term | Exp , '-' , Term | Term ;
    Term        = Term , '*' , Factor | Term , '/' , Factor | Factor ;
    Factor      = '~' , Value | Value ;
    Value       = Integer | Identifier ;
    Type        = "number" | "letter" ;
    Declare     = "was a" , Type ;
    Assign      = "became" , Exp | "became" , Char ;
    Return      = "said Alice" ;
    BuiltInFunc = "Ate" | "Drank" ;
    Code        = Identifier , Declare | Identifier , Assign | Exp , Return  
                | Identifier , BuiltInFunc ;
    Codeblock   = Code , Separator , Codeblock | Code , Terminator ; 
    Separator   = Terminator | ',' | "and" | "but" ; 
    Identifier  = String ;
    Terminator  = "too." | '.' | ',' ;

   - Primitives -
    String      = { Char } ;
    Integer     = '0' | [ '-' ] , Nat ;
    Nat         = DigitNoZero , { Digit } ;
    Digit       = '0' | DigitNoZero ; 
    DigitNoZero = '1' | '2' | '3' | '4' | '5' | '6' | '7' | '8' | '9' ;
    Char        = 'a' | 'b' | 'c' | 'd' | 'e' | 'f' | 'g' | 'h' | 'i' | 'j' | 'k' 
                | 'l' | 'm' | 'n' | 'o' | 'p' | 'q' | 'r' | 's' | 't' | 'u' | 'v'
                | 'w' | 'x' | 'y' | 'z' | 'A' | 'B' | 'C' | 'D' | 'E' | 'F' | 'G'
                | 'H' | 'I' | 'J' | 'K' | 'L' | 'M' | 'N' | 'O' | 'P' | 'Q' | 'R'
\end{verbatim}
\section{Semantics}
Operator Precedence
The operators in MAlice are defined below in increasing order of operator precidence.
\begin{itemize}
	\item
	$|$ Defines the logical bitwise OR operator. 
	\\ \tab e.g. 3 $|$ 5 = 7
	\item
	$\land$ Defines the logical bitwise XOR operator. 
	\\ \tab e.g. 3 $\land$ 5 = 6
	\item
	$\&$ Defines the logical bitwise AND operator. 
	\\ \tab e.g. 3 $\&$ 5 = 1
	\item
	+ - Define Integer Addition and Subtration respectively. 
	\\ \tab e.g. 3 + 5 = 8 \\ \tab e.g. 3 - 5 = -2
	\item
	* / \% Define Integer Multiplication, Division, and Modulo respectively. 
	\\ \tab e.g. 3 * 5 = 15 \\ \tab e.g. 3 / 5 = 0 \\ \tab e.g. 3 \% 5 = 3
	\item
	\textasciitilde  Defines the unary logical NOT operator.
	\\ \tab e.g. \textasciitilde$ 3 = -4$
    
    
\end{itemize}

\subsection{Illegal Operations}
The purpose of this section is to explain the behaviour of the edge cases of the
MAlice langugae.

\subsubsection{Divide by zero}
a devide by zero operations is going to be valid Alice syntax, as in the compiler
will allow for these statements and let the processor generate a runtime error.


\subsubsection{Numerical operators}
As MAlice is a stronly typed language, numerical operators can only have 
arguments of type number. any other type used with a numerical operator produces
a compile time type error this can be seen in the output for ex02.alice:


\subsubsection{Integer Overflow}
As integers are represented in an 8 bit two's compliment format there is a 
large chance for integer overflow to occur. Overflow should not be caught by the
compiler. it is up to the implementation on how to handle integer overflow, at 
runtime this language spec document will treat integer overflow as undefined 
behaviour.


\subsubsection{Variable Declaration}
variables in malice are declared by the BNF rules as above.
Malice is staticly typed, once a varible has been declared in a MAlice program 
it cannot then be redeclared. similarly you cannot introduce a variable without 
a type and expect the compiler to infer its type from an assignment statement for
example the follwing is invalid malice syntax

\texttt{wrong was a number.\\wrong was a letter.}

similarly variables cannot be re-declared even if they have the same type so the
following is also invalid.

\texttt{ wrong was a number then wrong was a number.}

Syntatic errors such as this are caught at compile time.


\subsubsection{Assignment}
Assignment follows the strict typing of MAlice, so a variable of type letter 
cannot take on a value of type number.


\subsubsection{Built in function}
Eating Drinking and talking: 




Finally for this document, if you want to include a reference
then you put it into a \texttt{thebibliography\{...\}}
environment (see below in source file) and then 
cite it like this \cite{lamport94}
(you will need to run \texttt{latex} twice to get it to process the citation),
or you can use BibTex but that is probably overkill for now.

\begin{thebibliography}{9}

\bibitem{lamport94}
  Leslie Lamport,
  \emph{\LaTeX: A Document Preparation System}.
  Addison Wesley, Massachusetts,
  2nd Edition,
  1994.
\bibitem{Wikipedia}
\begin{verbatim}	http://en.wikipedia.org/wiki/Extended_Backus%E2%80%93Naur_Form \end{verbatim}

\end{thebibliography}


\end{document}
