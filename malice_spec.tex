\documentclass[a4wide, 11pt]{article}
\usepackage{a4, fullpage}
\setlength{\parskip}{0.3cm}
\setlength{\parindent}{0cm}
\newcommand{\tab}{\hspace*{2em}}
% This is the preamble section where you can include extra packages etc.

\begin{document}

\title{MAlice Language Specification}

\author{Sean Allan \and Mitchell Allison \and Max Tottenham}

\date{\today}         % inserts today's date

\maketitle            % generates the title from the data above

\section{Introduction}

The goal of this document is to outline an informal language specification for 
the programming language MAlice. The following sections will outline the rough
gramatical structure of the language in Pseudo-Extended-BNF (PSEBNF). Then the 
document will explain the semantics, or meaning, of the logical and numerical 
operaters in malice.

\section{BNF Grammar} 

\begin{verbatim}
    - Malice Language Structure - 
    Function    = String ,  "()" ,  "opened" , Codeblock , "closed" ;
    LogExp      = LogExp , '&' , Exp |  LogExp , '|' , Exp | LogExp , '^' , Exp | Exp ;
    Exp         = Exp , '+' , Term | Exp , '-' , Term | Term ;
    Term        = Term , '*' , Factor | Term , '/' , Factor | Factor ;
    Factor      = '~' , Value | Value ;
    Value       = Integer | Identifier ;
    Type        = "number" | "letter" ;
    Declare     = "was a" , Type ;
    Assign      = "became" , Exp | "became" , Char ;
    Return      = "said Alice" ;
    Code        = Identifier , Declare | Identifier , Assign | Exp , Return ;
    Codeblock   = Code , Separator , Codeblock | Code , Terminator ; 
    Separator   = Terminator | ',' | "and" | "but" ; 
    Identifier  = String ;
    Terminator  = "too." | '.' | ',' ;
    

   - Primitives -
    String      = { Char } ;
    Integer     = '0' | [ '-' ] , Nat ;
    Nat         = DigitNoZero , { Digit } ;
    Digit       = '0' | DigitNoZero ; 
    DigitNoZero = '1' | '2' | '3' | '4' | '5' | '6' | '7' | '8' | '9' ;
    Char        = 'a' | 'b' | 'c' | 'd' | 'e' | 'f' | 'g' | 'h' | 'i' | 'j' | 'k' 
                | 'l' | 'm' | 'n' | 'o' | 'p' | 'q' | 'r' | 's' | 't' | 'u' | 'v'
                | 'w' | 'x' | 'y' | 'z' | 'A' | 'B' | 'C' | 'D' | 'E' | 'F' | 'G'
                | 'H' | 'I' | 'J' | 'K' | 'L' | 'M' | 'N' | 'O' | 'P' | 'Q' | 'R'
                | 'S' | 'T' | 'U' | 'V' | 'W' | 'X' | 'Y' | 'Z' | '_' ;
 
\end{verbatim}
The structure of a malice program:
	- main entry point
	- opening braces
	- defining variables
	- producing output/returning from main

\section{Semantics}
Operator Precedence
The operators in MAlice are defined below in increasing order of operator precidence.

\begin{itemize}
	\item
	$|$ Defines the logical bitwise OR operator. 
	\\ \tab e.g. 3 $|$ 5 = 7
	\item
	$\land$ Defines the logical bitwise XOR operator. 
	\\ \tab e.g. 3 $\land$ 5 = 6
	\item
	$\&$ Defines the logical bitwise AND operator. 
	\\ \tab e.g. 3 $\&$ 5 = 1
	\item
	+ - Define Integer Addition and Subtration respectively. 
	\\ \tab e.g. 3 + 5 = 8 \\ \tab e.g. 3 - 5 = -2
	\item
	* / \% Define Integer Multiplication, Division, and Modulo respectively. 
	\\ \tab e.g. 3 * 5 = 15 \\ \tab e.g. 3 / 5 = 0 \\ \tab e.g. 3 \% 5 = 3
	\item
	\textasciitilde  Defines the unary logical NOT operator.
	\\ \tab e.g. \textasciitilde$ 3 = -4$
\end{itemize}
TODO

\begin{enumerate}
    \item
    Explain operators 
    \item
    Explain legal operations    
    
    
\end{enumerate}

And here is a bulleted list.

\begin{itemize}

    \item
    The parts of the list are called items here too.
    
\end{itemize}

Finally for this document, if you want to include a reference
then you put it into a \texttt{thebibliography\{...\}}
environment (see below in source file) and then 
cite it like this \cite{lamport94}
(you will need to run \texttt{latex} twice to get it to process the citation),
or you can use BibTex but that is probably overkill for now.

\begin{thebibliography}{9}

\bibitem{lamport94}
  Leslie Lamport,
  \emph{\LaTeX: A Document Preparation System}.
  Addison Wesley, Massachusetts,
  2nd Edition,
  1994.
\bibitem{Wikipedia}
\begin{verbatim}	http://en.wikipedia.org/wiki/Extended_Backus%E2%80%93Naur_Form \end{verbatim}

\end{thebibliography}


\end{document}
