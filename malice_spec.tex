\documentclass[a4wide, 11pt]{article}
\usepackage{a4, fullpage}
\setlength{\parskip}{0.3cm}
\setlength{\parindent}{0cm}

% This is the preamble section where you can include extra packages etc.

\begin{document}

\title{MAlice Language Specification}

\author{Sean Allan \and Mitchell Allison \and Max Tottenham}

\date{\today}         % inserts today's date

\maketitle            % generates the title from the data above

\section{Introduction}

The goal of this document is to outline an informal language specification for 
the programming language MAlice. The following sections will outline the rough
gramatical structure of the language in Pseudo-Extended-BNF (PSEBNF). 

\section{BNF Grammar} 

\begin{verbatim}
	Function = String ,  "()" ,  "opened" , Codeblock , "closed" ;
	Exp = Exp '+' Term | Exp '-' Term | Term ;
	Term = Term '*' Factor | Term '/' Factor | Factor ;
	Factor = '~' Value | Value ;
	Value = Integer | Identifier ;
	Type = "number" | "letter" ;
	Declare = "was a" , Type ;
	Assign = "became" , Exp | "became" , Char ;
	Return = "said Alice" ;
	Code = 	Identifier , Declare | Identifier , Assign | Exp , Return ;
	Codeblock = Code , Separator , Codeblock | Code , Terminator ; 
	String = {Char} ;
	Separator = Terminator | ',' | "and" | "but" ; 
	Identifier = String ;
	
	Terminator = "too." | '.' | ',' ;
	Char = 'a' | 'b' | 'c' | 
	DigitNoZero =  '1' | '2' | '3' | '4' | '5' | '6' | '7' | '8' | '9' ;
	Digit = '0' | DigitNoZero ; 
	Nat = DigitNoZero , {Digit} ;
	Integer = '0' | [ '-' ] , Nat ;

\end{verbatim}

To include maths formulas in text put them between \$ symbols like this
$f(x) = x \times 5$.
Or to display a formula on a line on its own you do this:
\[
    g(y) = y^2
\]

\section{Semantics}

TODO LIST

\begin{enumerate}

    \item
    Figure out opperator precidence
    
    \item
    Make sure the gammar is left recursive
    
    \item
    Finish PSEBNF spec

    \item
    Use BNF with Scanning techniques 
    
    \item
     explain operators

    \item
    Explain legal operations    
    
    
\end{enumerate}

And here is a bulleted list.

\begin{itemize}

    \item
    The parts of the list are called items here too.
    
\end{itemize}

Finally for this document, if you want to include a reference
then you put it into a \texttt{thebibliography\{...\}}
environment (see below in source file) and then 
cite it like this \cite{lamport94}
(you will need to run \texttt{latex} twice to get it to process the citation),
or you can use BibTex but that is probably overkill for now.

\begin{thebibliography}{9}

\bibitem{lamport94}
  Leslie Lamport,
  \emph{\LaTeX: A Document Preparation System}.
  Addison Wesley, Massachusetts,
  2nd Edition,
  1994.
\bibitem{Wikipedia}
 

\end{thebibliography}


\end{document}
