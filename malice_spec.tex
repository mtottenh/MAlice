\documentclass[a4wide, 11pt]{article}
\usepackage{a4, fullpage}
\setlength{\parskip}{0.3cm}
\setlength{\parindent}{0cm}

% This is the preamble section where you can include extra packages etc.

\begin{document}

\title{MAlice Language Specification}

\author{Sean Allan \and Mitchell Alison \and Max Tottenham}

\date{\today}         % inserts today's date

\maketitle            % generates the title from the data above

\section{Introduction}

The goal of this document is to outline an informal language specification for 
the programming language MAlice. The following sections will outline the rough
gramatical structure of the language in Pseudo-Extended-BNF (PSEBNF). 

\section{BNF Grammar} 

\begin{verbatim}
    The verbatim environment outputs the source without changing
    it in any way. 
          This
              includes line breaks
       and indentation. 
    It is useful to reproduce code snippets.
\end{verbatim}

To include maths formulas in text put them between \$ symbols like this
$f(x) = x \times 5$.
Or to display a formula on a line on its own you do this:
\[
    g(y) = y^2
\]

\section{Semantics}

Here is a numbered list.

\begin{enumerate}

    \item
    This is item 1.
    
    \item
    And this is item 2.
    
\end{enumerate}

And here is a bulleted list.

\begin{itemize}

    \item
    The parts of the list are called items here too.
    
\end{itemize}

Finally for this document, if you want to include a reference
then you put it into a \texttt{thebibliography\{...\}}
environment (see below in source file) and then 
cite it like this \cite{lamport94}
(you will need to run \texttt{latex} twice to get it to process the citation),
or you can use BibTex but that is probably overkill for now.

\begin{thebibliography}{9}

\bibitem{lamport94}
  Leslie Lamport,
  \emph{\LaTeX: A Document Preparation System}.
  Addison Wesley, Massachusetts,
  2nd Edition,
  1994.

\end{thebibliography}


\end{document}
