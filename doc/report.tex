\documentclass[a4wide, 11pt]{article}
\usepackage{a4, fullpage}
\usepackage{graphicx}
\usepackage{verbatim}
\setlength{\parskip}{0.3cm}
\setlength{\parindent}{0cm}
\newcommand{\tab}{\hspace*{2em}}

\begin{document}

\title{MAlice Report}

\author{Sean Allan \and Mitchell Allison \and Max Tottenham}

\date{\today}         % inserts today's date

\maketitle            % generates the title from the data above

% TODO - Not sure if needed; spec doesn't mention an intro, and we're somewhat
% constrained on page length (max of 4!) 
\begin{comment}
\section{Introduction}
 
This report outlines the development of the group's compiler for the MAlice
language into x86-64 and ARM assembly code, from the design stages to 
implementation and extension. Throughout the project, careful attention was
given to design, the tools and language used and ensuring functional
correctness.
\end{comment}

% sections as given in the spec.
\section{The Product}

\section{The Design Choices}

When implementation began during Milestone 2, the group carefully considered
many options for lexical analysis and parser generation. The choice for making 
use of tools for this milestone was made due to the difficulty associated with
creating a parser generator for an LALR grammar. After looking at the
pros and cons of many packages, {\bf\emph{Flex}} was chosen for lexical analysis
and {\bf\emph{Bison}} chosen as the parser generator. This decision was made as 
these two packages are arguably the most stable and well documented tools for 
aiding in the creation of a compiler front end.

\section{Beyond the Specification}

\enddocument
