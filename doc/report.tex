\documentclass[a4wide, 11pt]{article}
\usepackage{a4, fullpage}
\usepackage{graphicx}
\usepackage{verbatim}
\usepackage[margin=1.0in]{geometry}
\setlength{\parskip}{0.3cm}
\setlength{\parindent}{0cm}
\newcommand{\tab}{\hspace*{2em}}

\begin{document}

\begin{center}
{\huge MAlice Report} \\ [0.4cm]
{\large Sean Allan, Max Tottenham and Mitchell Allison} \\ [0.2cm]
\vspace{1cm}
\end{center}

% sections as given in the spec.
\section{The Product}
The compiler the group has produced takes code written in the MAlice language
and produces x86-64 assembly code, as well as calling \emph{nasm} and \emph{gcc}
to produce an executable binary. If the compiler is ran with the \texttt{-arm} 
flag then, as part of the extension, an assembly file with extension 
\texttt{.arm} is produced containing code targetted at the ARM architecture.

The group ensured at every stage of development that the product produced
was functionally correct and matched the specification. After completion of
Milestone 2, the parser and semantic analysis components were tested against
all test cases, as well as some custom written test files, to ensure invalid
code raised the appropriate error messages. After completion of Milestone 3, it
was also ensured that the assembly code produced led to the correct result 
for most test cases, with the group fixing bugs and issues where appropriate.

Despite this, however, a minority of test cases 
(\texttt{bubbleSort.alice} and \texttt{vectorFunctions.alice}) do
not pass. The group made every effort to correct the bugs that caused these
tests to perform incorrectly and prevent segmentation faults, but due to time
constraints, this was not possible. Both of these cases fail due to an 
incomplete implementation of array handling. Additionally, as part of the
extension, the generated ARM code is not fully correct. This is discussed
in further detail in Section 3.

Due to decisions made in regard to design, discussed further in the next
section, the group ensured the product performed efficiently. The language
choice for the project, C++, helps to promote such efficiency, and the group
also made an effort to prevent memory leaks and free memory where required. 
Additionally, for each data structure used, the team considered what would be
best for the situation, such as choosing between a \emph{std::vector} or a
\emph{std::deque}. Finally, we made use of design patterns such as the visitor
pattern to aid further extensions and provide a solid base for future work.

\section{The Design Choices}

\subsection{Language Choice - C++}
The majority of the project was implemented using the C++ language. This
decision was made due to many advantages C++ offers - most of the speed and
efficiency of the C language with the addition of useful data structures such
as \emph{std::deque}s. The group did consider other languages, such as Java (as
recommended in the specification), but decided that the benefits of speed and
low level operations outweighed Java's advantages (e.g. garbage collection) in
this instance. A small amount of C is also used in order to better interface
with the Flex and Bison tools we used, described below.

\subsection{Parser - Flex and Bison}

When implementation began during Milestone 2, the group carefully considered
many options for lexical analysis and parser generation. The choice for making 
use of tools for this milestone was made due to the difficulty associated with
creating a parser generator for an LALR grammar. After looking at the
pros and cons of many packages, \emph{Flex} was chosen for lexical analysis
and \emph{Bison} chosen as the parser generator. This decision was made as 
these two packages are arguably the most stable and well documented tools for 
aiding in the creation of a compiler front end.

\subsection{Code Generation - Visitor Pattern}

For the code generation stage of the project, the visitor pattern was used, 
seperating the logic of code generation from the nodes contained within the
abstract syntax tree. For the generation of x86-64 code, the \texttt{x86CodeGenerator}
class calls \texttt{accept} on the root node of the abstract syntax tree, which
in turn calls the the visitor's \texttt{visit} function on itself (to match
the required specialisation of the node type). From there, the visitor decides
which nodes to visit next, generating code for each node in the correct order. 

To support future development, \texttt{x86CodeGenerator} extends a base class,
\texttt{CodeGenerator}, which in turn makes use of a generic \texttt{ASTVisitor}
As part of the extension, the \texttt{ARMCodeGenerator} class was
created that also extends this base class. The use of the visitor pattern
would greatly increase the ease of adding a new code generator for an additional
target language. This is discussed in further detail in Section 3.

\subsection{Linking with the C Standard Library}
Initially, the group made use of native Linux system calls for functions such
as printing to \texttt{stdout} and taking user input from \texttt{stdin}. 
However, this proved difficult and was prone to errors and bugs, and also
the group realised that this was detrimental to portability - the system calls
would be different on other platforms, such as Windows. Due to this, the 
decision was made to make use of C library functions. The MAlice compiler links,
via \emph{gcc}, with the C library and makes use of \texttt{printf} for output
and \texttt{scanf} for input. \texttt{malloc} is also used to allocate memory
on the heap for dynamically-allocated arrays, where the size of the array
cannot be known at compile time. 

Additionally, at the end of a MAlice assembly program, the C \texttt{exit} 
method is called rather than an exit system call. This ensures that 
\texttt{stdout} and \texttt{stderr} correctly have their buffers cleared at the
end of execution of the MAlice program.

\subsection{Future Work}

Although the group considered very heavily the design and how classes would
interact throughout Milestone 3, more could have been done during Milestone 2.
The group was perhaps a little too quick to begin writing code without 
considering how classes and objects would interact, and this led to the team 
running into problems that cost time. A future improvement to increase 
productivity would be to draw UML diagrams or class diagrams and to write
pseudocode in order to identify design issues before they occur in code.

Moreover, further consideration could be given to how MAlice stores data. 
Currently, for ease of code generation in x86-64, all variables are given
64 bits of storage, regardless of whether the variable contains a character,
number, or string. This is wasteful of memory at runtime. Additionally, in the
ARM extension (again, for ease of code generation), all variables are given
32 bits of storage. This introduces inconsistencies that may hinder a MAlice
developer who chose to use our compiler.

Finally, a potential area for future improvement could be to make use of
\emph{LLVM}. The use of this tool would allow the group to produce highly
optimised assembly code beyond the simple optimisations completed in the time
available. The use of LLVM would also give the group experience in a tool that
would likely be used in industry in compiler creation.

\section{Beyond the Specification}

\subsection{Optimisations}

\subsubsection{Binary Operators - Weight Function}

When a binary operator node has its code generated, a weight function is used
in order to enforce efficient use of registers. Each node is given a weight,
which represents the number of registers required to evaluate and store the
result of the computation. For example, constants and identifiers require a 
single register to store, hence they have a weight of 1. In a binary operator
expression, the weight function calculates how many registers will be required 
to evaluate each subexpression. It is then the job of the code generator to 
evaluate the expression with the largest weight first in order allow it one more
register than its companion.

\subsubsection{Tree Optimisations}

Another optimisation the group implemented was to identify unreachable MAlice 
methods. This is done prior to code generation in the abstract syntax tree,
and prevents unnecessary code being produced in the assembly file. This makes
the compiler more efficient, as well as leading to smaller executable file
sizes.

\subsection{Refactored Visitor Pattern and ARM Code Generation}

For the extension, the group split the \texttt{ASTVisitor} base class into two
seperate classes. It became apparent when considering extending the project to
support another target language that there would be a significant amount of 
code duplication between the visitor class for x86 and other platforms. 
Therefore, \texttt{ASTVisitor} was converted from an interface to a base class. 
The functionality that the group determined would differ between x86 and ARM 
(and potentially further target platforms) was moved to another base class, 
\texttt{CodeGenerator}.

To demonstrate this refactored visitor pattern, and the ease of implementing
cross-compilation, the group decided to compile MAlice programs into ARM
assembly code in addition to x86-64 code. This would allow MAlice developers
to theoretically write code that would run on a wide variety of additional
devices, such as Android smartphones and tablets. To do this, a new class, 
\texttt{ARMCodeGenerator} was implemented, that extends \texttt{CodeGenerator}.

Running the MAlice compiler with the \texttt{-arm} flag uses the abstract syntax
tree and the \texttt{ARMCodeGenerator} class to generate ARM code. However, unlike 
as with the x86-64 code generation step, the ARM code is not assembled or linked
with the C standard library, due to the lack of tools available for this on the
standard lab machine install. Despite this, members of the group installed the 
\texttt{arm-linux-gnueabihf} (providing versions of \emph{gcc} and \emph{as} for 
ARM) and \texttt{qemu-user} (providing a user mode ARM emulator) packages on 
their local machines to test the extension.

Unfortunately, it became apparent that the code generation step for an 
additional target language was not as simple as initially thought. ARM is a 
'load/store' language, and the way in which data is stored in memory (as opposed
to registers) is very different to x86. Additionally, the syntax differences
proved to be too large to deal with in the time available for the project. 
Despite this, ARM 'pseudocode' is produced and demonstrates as a proof of 
concept what would be possible with the \texttt{ASTVisitor}, 
\texttt{CodeGenerator} and a sufficient amount of time.

\subsection{Future Work: C Library Functions}

One extension that would make MAlice a much more powerful language for 
developers would be to introduce a language construct that allows a developer 
to call C library methods. For example, this would allow developers to use 
things such as file I/O, which are not supported natively in MAlice. Since 
MAlice code is already linked with the C standard library upon compilation
(as mentioned earlier, for \texttt{printf}, \texttt{scanf} and \texttt{exit}),
it would be relatively easy to implement this extension if additional time
were available.

\subsection{Future Work: Other Target Languages}
In a similar fashion to how the ARM extension was implemented, it would be 
possible to extend the compiler to produce machine code for other target 
languages using the visitor pattern and a class extending \texttt{ASTVisitor} -
for example, MIPS, Java bytecode or .NET CIL. 

One particularly interesting potential target language would be IMPS - a subset
of MIPS for which the group created an assembler and emulator for as part of the
first year C lab. This would mean the group had a complete set of tools for
each stage of compilation and execution of a MAlice program.

\enddocument
