\documentclass[a4wide, 11pt]{article}
\usepackage{a4, fullpage}
\usepackage{graphicx}
\usepackage{verbatim}
\setlength{\parskip}{0.3cm}
\setlength{\parindent}{0cm}
\newcommand{\tab}{\hspace*{2em}}

\begin{document}

\begin{center}
{\huge MAlice Report} \\ [0.4cm]
{\large Sean Allan, Max Tottenham and Mitchell Allison} \\ [0.2cm]
\vspace{1cm}
\end{center}

% sections as given in the spec.
\section{The Product}

The compiler the group has produced takes code written in the MAlice language
and produces x86-64 assembly code, as well as calling \emph{nasm} and \emph{gcc}
to produce an executable binary. Additionally, as part of the extension, an
assembly file with extension \texttt{.arm} is produced that can be assembled
to run on an ARM chip.

The group ensured at every stage of development that the product produced
was functionally correct and matched the specification. After completion of
Milestone 2, the parser and semantic analysis components were tested against
all test cases, as well as some custom written test files, to ensure invalid
code raised the appropriate error messages. After completion of Milestone 3, it
was also ensured that the assembly code produced led to the correct result 
for all test cases, with the group fixing bugs and issues where appropriate.

Due to decisions made in regard to design, discussed further in the next
section, the group ensured the product performed effeciently. The language
choice for the project, C++, helps to promote such effeciency, and the group
also made an effort to prevent memory leaks and free memory where required. 
Additionally, for each data structure used, the team considered what would be
best for the situation, such as choosing between a \emph{std::vector} or a
\emph{std::deque}. Finally, we made use of design patterns such as the visitor
pattern to aid further extensions and provide a solid base for future work.

\section{The Design Choices}

\subsection{Language Choice - C++}
% TODO - Maybe expand on this?
The majority of the project was implemented using the C++ language. This
decision was made due to many advantages C++ offers - most of the speed and
effeciency of the C language with the addition of useful data structures such
as \emph{std::deque}s. The group did consider other languages, such as Java (as
recommended in the specification), but decided that the benefits of speed and
low level operations outweighed Java's advantages (e.g. garbage collection) in
this instance.

\newpage

\subsection{Flex and Bison}

When implementation began during Milestone 2, the group carefully considered
many options for lexical analysis and parser generation. The choice for making 
use of tools for this milestone was made due to the difficulty associated with
creating a parser generator for an LALR grammar. After looking at the
pros and cons of many packages, \emph{Flex} was chosen for lexical analysis
and \emph{Bison} chosen as the parser generator. This decision was made as 
these two packages are arguably the most stable and well documented tools for 
aiding in the creation of a compiler front end.

\subsection{Code Generation - Visitor Pattern}

For the code generation stage of the project, the visitor pattern was used, 
seperating the logic of code generation from the nodes contained within the
abstract syntax tree. For the generation of x86-64 code, the \texttt{x86Visitor}
class calls \texttt{accept} on the root node of the abstract syntax tree, which
in turn calls the the visitor's \texttt{visit} on its children, and so on, 
generating code for each node in order. 

To support future development, \texttt{x86Visitor} implements an interface,
\texttt{ASTVisitor}. As part of the extension, the \texttt{ARMVisitor} class was
created that also implements this interface. The use of the visitor pattern
made creating this additional visitor quick and easy - and due to the lessons
learnt from \texttt{x86Visitor}, the \texttt{ARMVisitor} becamse mostly a
matter of translating syntax from x86-64 to ARM. Additional visitors, for other
assembly languages, would also be made easier through this design pattern.

\subsection{Future Work}



\section{Beyond the Specification}

\enddocument
